\documentclass{report}
\usepackage[T1]{fontenc} % Fontes T1
\usepackage[utf8]{inputenc} % Input UTF8
\usepackage[backend=biber, style=ieee]{biblatex} % para usar bibliografia
\usepackage{csquotes}
\usepackage[portuguese]{babel} %Usar língua portuguesa
\usepackage{blindtext} % Gerar texto automaticamente
\usepackage[printonlyused]{acronym}
\usepackage{hyperref} % para autoref
\usepackage{graphicx}
\usepackage{indentfirst}
\graphicspath{{imagens/}}
\bibliography{bibliografia.bib}

\begin{document}

%%
% Definições
%
\def\titulo{Misturador de Músicas}
\def\data{29/05/2018}
\def\autores{André Alves, Daniel Correia, Pedro Almeida, Pedro Valente}
\def\autorescontactos{(88811) andr.alves@ua.pt, (88753) dcorreia@ua.pt, (89205) pedro22@ua.pt, (88858)pedro.valente@ua.pt}
\def\departamento{Departamento de Electrónica, Telecomunicações e Informática}
\def\empresa{Universidade De Aveiro}
\def\logotipo{img/ua.pdf}
%

%%%%%% CAPA %%%%%%
%
\begin{titlepage}

\begin{center}
%
\vspace*{50mm}
%
{\Huge \titulo}\\ 
%
\vspace{10mm}
%
{\Large \empresa}\\
%
\vspace{10mm}
%
{\LARGE \autores}\\ 
%
\vspace{30mm}
%
\begin{figure}[h]
\center
\includegraphics{\logotipo}
\end{figure}
%
\vspace{30mm}
\end{center}
%

\end{titlepage}

%%  Página de Título %%
\title{%
{\Huge\textbf{\titulo}}\\
{\Large \departamento\\ \empresa}
}
%
\author{%
    \autores \\
    \autorescontactos
}
%
\date{\data}
%
\maketitle

\pagenumbering{arabic}


\tableofcontents
\listoffigures    % descomentar se necessário



\chapter{Introdução}
\label{chap.introducao}
Este relatório tem como propósito explicitar como o projeto foi desenvolvido, passando desde as especificações do alarme, a sua arquitetura, manual de instruções e a divisão das tarefas pelos elementos do grupo.

O projeto consistiu em implementar um módulo de controlo para um sistema de alarme doméstico. O alarme permite dois modos de disparo do alarme, modo interno e modo externo. No modo interno apenas os sensores das janelas, dois dos sensores de presença e o botão de pânico ativam o alarme. Já no modo externo, todos os sensores ativam o alarme. O alarme permite também a sua ativação rápida recorrendo a um botão de pânico. Para desativar o alarme é necessário introduzir um código secreto sendo que, ao fim de três tentativas falhadas, o alarme disparará. É também possível ver as últimas ocorrências que fizeram com que o alarme fosse disparado.

No \autoref{chap.especificações} serão apresentadas as características mais especificas do alarme. No \autoref{chap.arquitetura} será apresentada a arquitetura detalhada do alarme assim como o esquema da máquina de estados. O \autoref{chap.M_Instr} diz respeito ao Manual de Instruções do alarme, primeiramente de uma fase inicial e seguidamente de uma fase mais avançada do alarme. Posteriormente, no \autoref{chap.conclusao} serão apresentadas as conclusões finais do projeto. Finalmente, no \autoref{contribuições} será explicitado as contribuições de cada autor.
	



\chapter{Especificações}
\label{chap.especificações}
	
As especificações do alarme são:
	\begin{itemize}
		\item Depois do alarme ser ativado, o utilizador tem 20 segundos para sair de casa;
		
		\item O alarme ligado é sinalizado pelo piscar do LEDG(0) à frequência de 1Hz;
		
		\item Após a ativação de um dos sensores de presença, o utilizador possui dez segundos para desligar o alarme, caso contrário o alarme dispara. Se um dos sensores das janelas for ativado, o alarme dispara imediatamente;
		
		\item Os interruptores SW(3..0) representam os sensores das janelas;
		
		\item Os interruptores SW(7..4) representam os sensores de presença no interior da habitação; 
		
		\item As temporizações são todas indicadas nos displays de 7 segmentos;
		
		\item O modo do alarme é controlado pelo SW(10) e é indicado no LCD;
		
		\item O botão de pânico é o SW(8);
		
		\item A sirene do alarme corresponde ao piscar intermitente  de vários LEDR;
		
		\item  O alarme é desativado por dois códigos secretos distintos, sendo um fixo (sequência "1,2,3,4" na fase básica e "lsdfixe" na fase mais avançada) e um outro programável;
		
		\item Ao inserir o código secreto vai aparecendo um "*" no LCD por cada carácter inserido;
		
		\item O código secreto pode ser introduzido através de um teclado ps2;
		 
	\end{itemize}

	

\chapter{Arquitetura}
\label{chap.arquitetura}

Na figura 3.1 é apresentada a arquitetura geral do sistema de alarme doméstico. Assim como na figura 3.2, está representada a máquina de estados do sistema.

	\begin{figure} [t]
		\centering
		\includegraphics[scale=0.6, angle = 90]{img/arq_final.PNG}
		\caption{Arquitetura geral}
	\end{figure}

	\begin{figure} [h]
		\centering
		\includegraphics[scale=0.7]{img/m_estados.PNG}
		\caption{Máquina de Estados}
	\end{figure}

	
	
\chapter{Manual de instruções}
\label{chap.M_Instr}
	
	Na figura 4.1 apresenta-se o manual de instruções, seguido de os passos de modo a utilizar o alarme numa fase básica do projeto.  
	
	\begin{figure} [h]
		\centering
		\includegraphics[scale=0.65]{img/m_instruc_ini.PNG}
		\caption{Manual de instruções - modelo básico}
	\end{figure}

\newpage
	
	\textbf{Como funciona:}
	\begin{enumerate}
 		\item Ligar o alarme:\\
 		Premir botão ligar (Tecla4).
 		
 		\item Botão de pânico:\\
 		Ligar interruptor botão de pânico. 
 		
 		\item Registo das últimas ocorrências:\\
 		Premir botão memória. A informação aparece no display de registo. 
 		A informação é transmitida da seguinte forma:
 		1..7 - Sensor que fez disparar alarme;
 		8 - botão de pânico;
 		9 - inseriu código secreto errado três vezes. 	
 		Premir botão voltar para sair do registo.
 		
 		\item Alterar código secreto:\\
 		Premir Tecla 3 seguido de Tecla 4 e depois o novo código secreto.
 		
 		\item Desligar alarme:\\
 		Introduzir código secreto. 		
	\end{enumerate}	
	
	
	
	 Na figura 4.2 apresenta-se o manual de instruções, seguido de os passos de modo a utilizar o alarme na fase mais avançada do projeto.
	
	\begin{figure} [h]
		\centering
		\includegraphics[scale=0.65]{img/m_instruc_avan.PNG}
		\caption{Manual de instruções - modelo avançado}
	\end{figure}
	
\newpage

	\textbf{Como funciona:}
	\begin{enumerate}
 		\item Ligar o alarme:\\
 		Premir botão ligar (Tecla4).
 		
 		\item Definir modo interno/externo:\\
 		Alternar interruptor modo interno/externo. O modo em que o alarme                    	    se encontra é apresentado no LCD de informações diversas.
 		
 		\item Botão de pânico:\\
 		Ligar interruptor botão de pânico. 
 		
 		\item Registo das últimas ocorrências:\\
 		Premir botão memória. A informação aparece no LCD de informações 			diversas. 
 		A informação é transmitida da seguinte forma:
 		1..7 - Sensor que fez disparar alarme;
 		8 - botão de pânico;
 		9 - inseriu código secreto errado três vezes.
 		Premir botão voltar para sair do registo.
 		
 		\item Alterar código secreto:\\
 		Premir Tecla 3 seguido de Enter e depois o novo código secreto.
 		
 		\item Desligar alarme:\\
 		Introduzir código secreto. 		
	\end{enumerate}
	
	
	Nesta versão do alarme, durante a sua utilização, serão também apresentadas no LCD de informações diversas alguns dos passos explicitados anteriormente, de forma a ser mais intuitivo.
	
	 

	
\chapter{Conclusões Finais}
\label{chap.conclusao}
	Com este projeto de sistema de alarme doméstico foi possível consolidar os conhecimentos adquiridos em aulas pois foi necessário tudo o que foi aprendido durante o semestre. 
	Foi também possível, não só implementar e por a funcionar cada módulo singularmente, mas ver todos a funcionar em conjunto para um propósito. 

 
\chapter{Contribuições dos autores}
\label{contribuições}

\noindent
Daniel Correia - 60\% \\
Pedro Almeida - 40\%


\end{document}
